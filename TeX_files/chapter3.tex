% !TeX spellcheck = en_GB
\chapter{Extension to constant curvature spaces}
In this chapter we want to extend the Chow-Gulliver result (theorem \ref{chow gulliver}) to constant curvature spaces. We will use the same notation as in section 
\ref{reflections definitions} and \ref{method moving planes}. 
Throughout the chapter, by hyperplane we mean a totally geodesic hyperplane. 

\section{The equation in constant curvature spaces}	
As shown in \cite{huisken}, the equation we analysed in the previous chapter can be also considered in non-flat ambient spaces. In particular, we will analyse the case where the ambient space is one of those described  in section \ref{reflections definitions}: $\R^{n+1}$, $\mathbb{H}^{n+1}$, or $\mathbb{S}^n_+$. We again use the symbol $\mathbb{M}^{n+1}_+$ to indicate any of these spaces. 
Let $X_0 : M^n \rightarrow \mathbb{M}^{n+1}_+$ be a manifold embedded in $\mathbb{M}^{n+1}_+$. 
Let $F:\{(\kappa_1, \dots , \kappa_n)\in \R^n\vert \kappa_1\leq \dots \leq \kappa_n\}\rightarrow \R$ be a $C^1$ function satisfying:
\begin{equation}
	\frac{\partial F}{\partial \kappa_i} > 0 \mathrm{\; for \; all } \; i=1,\dots, n \label{parabolicità_ext}
\end{equation}
and consider the evolution equation 
\begin{align}
	\begin{dcases}
		\frac{\partial X_t}{\partial t} = - F(\kappa_1(x), \dots , \kappa_n(x)) \nu\\
		X(0)= X_0
	\end{dcases} \label{evolutioneq_ext}
\end{align}
where $\nu$ is the outward normal to $X_t(M^n)$ at the point $X_t(x)$ and $\kappa_1\leq \dots \leq \kappa_n$ are the principal curvatures at $X_t(x)$. 

As we saw in the previous chapter, it is a non-linear parabolic differential equation, as the calculation in chapter \ref{parabolic} is valid for any metric on $\R{n+1}$, and in particular, for the metrics in our models for $\mathbb{M}^{n+1}_+$ in section \ref{reflections definitions}. The existence result in \cite{huisken} also holds as well in this case. Finally, the result in section \ref{representation as graph} is not using the metric tensor, and therefore is valid in this setting as well, again because there are models on $\R^{n+1}$. 

\section{Extension of the result}	


Assume the hypothesis in section \ref{reflections definitions}: $X:M^n\rightarrow \mathbb{M}^{n+1}_+$ is a hypersurface in a constant curvature ambient space, and we choose a point and a direction $v$ to foliate the ambient space. Consider a hyperplane in the foliation, $\pi=\pi_{v, C}$. As in section \ref{reflections definitions} we can define the reflection about $\pi$. As in the previous chapter, let $X^\pi$ be the reflection of X about $\pi$.

Then, $\mathbb{M}^{n+1}_+$ is divided by $\pi$ into two half-spaces:
\begin{align*}
	H^+(\pi)= \bigcup_{s>C} \pi_{v, s} \;\;\mathrm{and}\;\;	H^-(\pi)= \bigcup_{s<C} \pi_{v, s}.
\end{align*} 

\begin{defin}
	We say {\em we can reflect $X: M^n\rightarrow \mathbb{M}^{n+1}_+$ strictly with respect to $\pi$} if both:
	\begin{itemize}
		\item $X^\pi\cap H^-(\pi)\subset \mathrm{int}(X)\cap H^-(\pi)$ 
		\item The tangent spaces $T_xX$ and $T_xX^\pi$ do not coincide for each $x\in X(M^n) \cap \pi$ (when seen as subspaces of  $T_x\mathbb{M}^{n+1}$)
	\end{itemize} 
\end{defin}
This fundamentally means that the reflection of one of the halves of $X$ on the other side of $\pi$ is contained in the region inside $M^n$ and the tangent spaces of $X$ and of the half-reflection do not form a ninety degree angle with $\pi$, at all points on $\pi\cap X$. As the two tangent spaces are one the reflection of the other, this means that they do not coincide.   
\begin{defin}
	We say {\em we can reflect $X: M^n\rightarrow \R^{n+1}$ strictly up to $(\pi,v)$} if we can reflect $M^n$ strictly with respect to $\pi_{v, s}$ for all hyperplanes $\pi_{v, s}$ such that $s<C$.  
\end{defin}

The result then becomes:


\begin{theorem}[Extended Chow-Gulliver]\label{chow gulliver extended}
	Let $X:M^n\times [0,T) \rightarrow \R^{n+1}$ be a $C^2$ solution to equation (\ref{evolutioneq_ext}). Then, if we can reflect $X(M^n, 0)=M_0$ strictly with respect to $\pi$, then for all $t\in [0,T)$ we can reflect $X(M^n, t)=M_t$ strictly with respect to $\pi$. 
\end{theorem}

\begin{proof}
	As before, by contradiction, suppose that there is a time $t$ such that the thesis is false, and that it is the smallest such $t$. Then, for all $\tau \in [0,t)$, $M_{\tau,\pi}\cap H^-(\pi)\subset \mathrm{int}(M_{\tau})\cap H^-(\pi)$; the unit vector orthogonal to $\pi$, $V$, is such that $V\notin T_xM_\tau$ for all $x\in M_\tau\cap \pi$ and $\tau \in [0,t)$; and either of the conditions fails at $t$, i.e. either: 
	\begin{itemize}
		\item[(i)] $M_{t,\pi}\cap H^-(\pi)\cap M_{t}\neq \emptyset$
		\item[(ii)] The tangent spaces $T_xX$ and $T_xX^\pi$ coincide for some $x\in\pi$. 
	\end{itemize} 
	
	(i) Suppose the first case is true. Then, there exists $x_0 \in M_{t,\pi}\cap H^-(\pi)\cap M_{t}$ such that at $x_0$ the two manifolds are tangent. \\
	We can then reason as in the proof of theorem \ref{chow gulliver}.
	
	(ii) Suppose instead that  $T_xM_t= T_xM_t^\pi$ and in a neighbourhood of $(x, t)$ both $M_t$ and $M_t^\pi$ are graphs of two smooth functions over $T_xM_t$ by \ref{localgraph}, i.e. again
	\begin{align*}
		f \; : \; (x, t) &\mapsto x+\tilde{f}(x, t)\nu \\
		f_\pi \; : \; (x, t) &\mapsto x+\tilde{f}_\pi(x, t)\nu 
	\end{align*} 
	We can do this because the theorem allowing us to do so is a theorem on smooth manifolds, and requires nothing on the metric, therefore the existence of models in $\R^{n+1}$ of the ambient spaces allows us to do the same procedure. 
	Reasoning again as in the proof of theorem \ref{chow gulliver}, in $\overline{H^-(\pi)}$, $f_\pi\geq f$, because $M^n_\pi\cap H^-(\pi)\subset \mathrm{int}(M^n)\cap H^-(\pi)$. Finally, $f(x, t)=f_\pi (x, t)$, hence $f_\pi-f (x, t)=0$, and thus  $(x, t)$ is a minimum point on the boundary for $f_\pi-f$. Also, for the outward pointing normal to the boundary $V$
	\begin{align*}
		\frac{\partial f}{\partial V}(x,t)=\frac{\partial f_\pi}{\partial V}(x,t)
	\end{align*}
	because the graphs are both tangent to $T_xM_t$. We note also that the models in section \ref{reflections definitions} are conformal, so a vector is orthogonal to the boundary if and only if it is orthogonal to the corresponding region when seen as a region in $\R^{n+1}$. Thus, 
	\begin{align*}
		\frac{\partial (f- f_\pi)}{\partial V}(x,t)=0
	\end{align*}
	But we must have 
	\begin{align*}
		\frac{\partial (f- f_\pi)}{\partial V}(x,t)>0
	\end{align*}
	at a minimum on the boundary by Proposition \ref{secondapplication}, a contradiction.  
\end{proof}
\section{Extending Corollaries}

We now shift the focus to extending the corollaries of theorem \ref{chow gulliver} in section \ref{Some corollaries of the result}. 

The author is not aware of a standard definition of a support and a radial function in a curved setting, so we do not attempt to extend the results in section \ref{Applying the result to find gradient estimates}. { \textbf{[DA RIMUOVERE SE CAMBIAMO IDEA]}}

Clearly, corollary \ref{upto} has a direct equivalent in this setting:
\begin{cor}
	Let $X:M^n\times [0,T) \rightarrow \mathbb{M}^{n+1}_+$ be a $C^2$ solution to equation (\ref{evolutioneq}). Then, if we can reflect $X_0$ strictly up to $(\pi_{v,C},v)$, for all $t\in [0,T)$ we can reflect $X_t$ strictly up to $(\pi_{v,C},v)$.  
\end{cor}

The second result that we can extend is Corollary \ref{graph}, although the meaning of \textit{graph} in curved spaces is ambiguous. We adapt the corollary as follows: 


\begin{cor}
	Let $X:M^n\times [0,T) \rightarrow \mathbb{M}^{n+1}_+$ be a $C^2$ embedded solution to equation (\ref{evolutioneq_ext}). Then, if we can reflect $X_0$ strictly up to $(\pi_{v,C},v)$, then for all $t\in [0,T)$, $X_t \cap H^+(\pi_{v,C})$ is such that the projection of the coordinates of its points onto $\pi_{v,C}$, $(p, \tau)\mapsto p$, is injective. 
\end{cor}
This condition guarantees that we can build a map from $s:\overline{\mathrm{int}(X_t \cap \pi_{v,C})}\rightarrow \R$ such that $(p, s(p))\in X_t$, making $p\mapsto (p, s(p))$ act as a sort of curved graph. 
\begin{proof}
	By  theorem \ref{chow gulliver extended} we can reflect up to $(\pi_{v,C},v)$ at all times in $[0,T)$. Let $\gamma_p(s)=(p, s+C)$ be the path followed by $p\in \pi_{v,C}$ as the planes sweep through the ambient space. If two points exist with the same $p$ coordinate, say $(p, C_1)$ and $(p, C_2)$, the reflection about the hyperplane $\pi_{v,\frac{ C_1 +  C_2}{2}}$ would map one onto the other, but as both $C_1$ and $C_2$ are greater than $C$, then we should also be able to reflect strictly about it, as we would have $\frac{ C_1 +  C_2}{2}>C$. 
\end{proof}

Let $\tilde{B}_r(y) = \{ p \in \mathbb{M}^{n+1}_+  : \mathrm{dist}(p, y) < r \}$. Reasoning exactly like in corollary \ref{reflect a small bit}, we also can extend it to this setting:  
\begin{cor}
	Let $X:M^n\times [0,T) \rightarrow \mathbb{M}^{n+1}_+$ be a $C^2$ embedded solution to equation (\ref{evolutioneq_ext}). There exists $\varepsilon>0$ depending only on $X_0$ such that for all $t\in[0, T)$ we can reflect $X_t$ up to $(\Pi_0^v +\epsilon v, v)$ for every $v \in S^n$. In particular, if $X_0 \subset \tilde{B}_R(C)$, then we can always reflect $X_t$ up to $(\Pi, v)$ whenever $H^+(\Pi)\cap \tilde{B}_{R-\varepsilon}(C)=\emptyset$.\label{reflect a small bit ext}
\end{cor}

{\Large \textbf{[Corollary \ref{x projection estimate} - If/when it can be done  - DA SCRIVERE]}}

{\Large \textbf{[Corollary \ref{sandwich estimate} - If/when it can be done  - DA SCRIVERE]}} I fear that this one cannot be done because the line mapping the minimum into the maximum in the reflection is not a geodesic anymore in the hyperbolic case


\section{Extending Ancient solutions section}

{\Large \textbf{[DA SCRIVERE]}}

Like before: 
\begin{defin}
	We say that a solution to \ref{evolutioneq_ext} is {\em expansive} if $F<0$. 
\end{defin}
\begin{defin}
	Let $ X : M^n \times (T_0, T_1) \to \mathbb{M}^{n+1}_+ $ be an embedded expansive solution to equation (\ref{evolutioneq_ext}). We say that {\em $X$ comes out of a point} if there exists a point $y_\infty$ such that for every $\varepsilon>0$, there exists a time $\tau \in  (T_0, T_1)$ such that $X_\tau \subset \tilde{B}_\varepsilon(y_\infty)$, where $\tilde{B}_r(y) = \{ p \in \mathbb{M}^{n+1}_+  : \mathrm{dist}(p, y) < r \}$.
\end{defin}

{\Large \textbf{[DA SCRIVERE]}}


\begin{theorem}
	Let $ X : M^n \times (T_0, T_1) \to \mathbb{M}^{n+1}_+ $ be a smooth, closed, embedded expansive solution to equation (\ref{evolutioneq_ext}) coming out of a point. Then it is a family of expanding distance spheres.
\end{theorem}


\begin{proof}
	{\Large \textbf{[DA CONTROLLARE]}}
	
	Fix any hyperplane $\pi$ not passing through $y_\infty$. There is $R>0$ such that $ \tilde{B}_{2R}(y_\infty)$ does not intersect it. By definition, there is also a time  $\tau \in  (T_0, T_1)$ such that $X_\tau \subset \tilde{B}_R(y_\infty)$. 
	By Corollary \ref{reflect a small bit ext}, then, we can reflect strictly $X_t$ up to $\pi$ for any $t>\tau$. 
	
	Now consider a sequence  $\epsilon_n\rightarrow 0$. Up to a subsequence, we can then find a corresponding converging non-increasing sequence $\tau_n\rightarrow \overline{t}\in [T_0, T_1)$ (here note that $\overline{t}$ can be $-\infty$) such that $X_{\tau_n}\subseteq B_{\epsilon_n}(y_\infty)$, therefore in $(\tau_n, T_1)$ I can reflect up to any hyperplane outside  $\tilde{B}_{\epsilon_n}(y_\infty)$ in any direction, reasoning like we just did. At time $\overline{t}$, $X_{\overline{t}} \subseteq \cap_r  \tilde{B}_{r}(y_\infty) = \{y_\infty\}$, thus we would have a singularity at $\overline{t}$ if  $\overline{t}\in(T_0, T_1)$ and therefore $\overline{t}=T_0$. 
	
	On the other hand, by construction, we can reflect $X_{t}$ strictly about any hyperplane not intersecting $\cap_r B_{r}(y_\infty) = \{y_\infty\}$ at any time $t>\overline{t}=T_0$, hence we can reflect $X_{t}$ strictly up to any hyperplane passing through $y_\infty$, in both directions, at any  time $t\in (T_0, T_1)$. We observe that in the limit, the reflection property becomes non-strict, in the sense that we have to replace the interior of $X$ in definition \ref{strict-reflection-definition} with its closure, therefore $X_{t}$ may touch its reflection at the limit plane, i.e. the one passing through $y_\infty$. Similarly, it cannot be that the other condition is the one causing the strict reflection definition to fail, as the other condition stays instead strict. 
	
	This implies that, taking any hyperplane passing through $y_\infty$ and considering opposite directions for the reflection, $X_{t}$ is symmetric about said hyperplane for any time $t\in (T_0, T_1)$. By Proposition \ref{proposition symmetry conclusion}, then, we conclude that $X_{t}$ must be a ball.	
\end{proof}
