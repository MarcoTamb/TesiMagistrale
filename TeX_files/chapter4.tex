% !TeX spellcheck = en_GB
\chapter{Area- and volume-preserving flows}


In this chapter area-preserving and volume-preserving flows are discussed. These are flows similar to the equation (\ref{evolutioneq}), with a non-local term added. It can be shown that the theorems in chapter 2 also apply to these flows. Finally, we discuss an application of these flows. 
\section{The flows we consider in this chapter}\label{Area- and volume-preserving flows}
Volume-preserving flows were introduced by Gerhard Huisken in 1987 (see \cite{volpres}). Unlike the mean curvature flow, which contracts hypersurfaces to a point, volume-preserving flows maintain a constant enclosed volume while reducing the hypersurface area. These flows are governed by the modified evolution equation:
\begin{align*}
	\frac{\partial X}{\partial t} = \left[h(t) - H(x, t)\right]\nu,
\end{align*}
where $ H(x, t)$ is the is the mean curvature at a point $x\in X_t$ and $h(t)$ is the average mean curvature at time $t$, a non-local term defined as 
\begin{align*}
	h(t) = \frac{\int_{X_t} H(x, t) \, d\mu}{\int_{X_t} 1 \, d\mu}= \frac{\int_{M_t} H(x, t) \, d\mu}{|M_t|},
\end{align*}
where $d\mu$ is the volume element on the hypersurface \( X_t \).
Notice that mean-curvature flows ($\frac{\partial X}{\partial t} = - H(t)\nu$) is a possible choice of $F$ in (\ref{evolutioneq}), therefore compared to the previous chapters we are just adding the global term $h(t)$. Huisken's results demonstrate that such flows lead to the convergence of the hypersurfaces to round spheres, provided the initial hypersurface is uniformly convex. 

A more general family of flows is in \cite{calcvar15}. Consider flows: 
\begin{align}
	\begin{dcases}
		\frac{\partial X_t}{\partial t} = \left[- H^k(x, t) + \phi(t)\right] \nu\\
		X(0)= X_0
	\end{dcases} \label{chapter 4 equation}
\end{align}
where $\phi(t)$ is an appropriate non-local term and $k\in (0, \infty)$. If $k\neq 1$, we will assume that  $H>0$ everywhere on $X_t$ for all $t$ (see remark below). 
\begin{oss}\em
	When considering $\frac{a+b}{2}$, the derivatives with respect to $a$ and $b$ are $\frac{1}{2}>0$. However, when one considers $ \left(\frac{a+b}{2}\right)^2=\frac{a^2}{4} + \frac{ab}{2} +\frac{b^2}{4} $,
	the derivatives depend on the values of $a$ and $b$, and in particular it may be negative depending on their values. $a$ and $b$ here can be our mean curvatures, and therefore this simple observation shows that there is no guarantee that the flow 
	\begin{align*}
		\begin{dcases}
			\frac{\partial X_t}{\partial t} = - H^k(x, t)  \nu\\
			X(0)= X_0
		\end{dcases} 
	\end{align*}
	is parabolic. One needs stronger hypothesis to guarantee this. By the formula for the derivative of the composite function, one gets easily
	\begin{align*}
		\frac{\partial }{\partial \kappa_i}  H^k =  k H^{k-1} \frac{\partial }{\partial \kappa_i}  H = \frac{k}{n} H^{k-1}
	\end{align*}
	To guarantee that $\frac{\partial F}{\partial \kappa_i} > 0$ we need to assume that $H>0$. We will assume this on the hypersurface at all times whenever $k\neq 1$.
\end{oss}
A possible choice for $\phi$ is: 
\begin{align}
	\phi(t) = \frac{\int_{X_t} H^k(x, t) \, d\mu}{|X_t|}, \label{phi-volume-preserving}
\end{align}
corresponding to volume-preserving flows. Another possible choice is
\begin{align}
	\phi(t) = \frac{\int_{X_t} H^{k+1}(x, t) \, d\mu}{\int_{X_t} H(x, t) \, d\mu}, \label{phi-area-preserving}
\end{align}
which corresponds to area-preserving flows. We know that, indeed, if we choose $\phi$ as in (\ref{phi-volume-preserving}), the volume of the domain enclosed by $X_t$ remains constant, while choosing $\phi$ as in (\ref{phi-area-preserving}) keeps the area $|X_t|$ constant. 

The fact that choosing $\phi$ as in (\ref{phi-volume-preserving}) preserves the volume is immediately apparent, as the change of the volume inside the hypersurface is the integral over $X_t$ of $\frac{\partial X_t}{\partial t}\cdot \nu$ (by Reynolds transport theorem for a constant function): 

\begin{align*}
		\int_{X_t}\frac{\partial X_t}{\partial t}\cdot \nu \;d\mu= \int_{X_t} \left[- H^k(x, t) + \phi(t)\right] d\mu = - \int_{X_t}  H^k(x, t) d\mu + \phi(t)|X_t|=0
\end{align*}
On the other hand, choosing $\phi$ as in (\ref{phi-area-preserving}), the formula for the first variation of area says that 
\begin{align*}
	\frac{d}{d t}\int_{X_t} d\mu = \int_{X_t} \left\langle\frac{\partial X_t}{\partial t} , H \nu \right\rangle d\mu  +\cancel{\int_{\partial X_t}\iota_{\frac{\partial X_t}{\partial t}}  d\mu}
\end{align*}
where the second term is cancelled because $\frac{\partial X_t}{\partial t}$ is orthogonal to the surface. Therefore
\begin{align*}
	\frac{d}{d t}\int_{X_t} d\mu &= \int_{X_t}  \left[- H^k + \phi(t)\right] H d\mu \\
	&= \phi(t)\int_{X_t} H d\mu  - \int_{X_t}  H^{k+1} d\mu  = 0
\end{align*}

\begin{oss}\em
Another possible choice for an area-preserving flow is in \cite{mccoy} where he considers a flow 
\begin{align}
	\begin{dcases}
		\frac{\partial X_t}{\partial t} = \left[1 - H(x, t)  \psi(t)\right] \nu\\
		X(0)= X_0
	\end{dcases} \label{mccoy equation}
\end{align}
where 
\begin{align*}
	\psi(t) = \frac{\int_{X_t} H(x, t) \, d\mu}{\int_{X_t} H^{2}(x, t) \, d\mu}
\end{align*}
Again, computing the first variation of the area:
\begin{align*}
	\frac{d}{d t}\int_{X_t} d\mu &= \int_{X_t}  \left[1 - H\phi(t)\right] H d\mu \\
	&= \int_{X_t}  H d\mu  - \psi(t)\int_{X_t} H^{2} d\mu = 0
\end{align*}
Notice, however, that it is almost the same flow as (\ref{chapter 4 equation}) with $k=1$, where we divided the RHS by $\phi(t)$.
\end{oss}
\section{Theorem \ref{chow gulliver} and corollaries}\label{Theorem CG and corollaries}
While not really parabolic flows, because of the non-local term, it is possible to apply theorems on parabolic flows to equation (\ref{chapter 4 equation}). Given a solution $X_t$, we can consider the evolution equation 
\begin{align}
	\begin{dcases}
		\frac{\partial Y_t}{\partial t} = \left[- H^k(x, t) + \varphi_{X_t}(t)\right] \nu\\
		Y(0)= Y_0
	\end{dcases} \label{parabolic version}
\end{align}
where $\varphi_{X_t}$ is the constant function independent of $Y$ given by $\phi$ when computed on the specific solution $X_t$. 
Clearly, $X_t$ is a solution also to the second equation, as the values of  $\phi$ and $\varphi_{X_t}$ coincide when one considers the solution $X_t$. At the same time, equation (\ref{parabolic version}) is a parabolic equation: the only difference with (\ref{evolutioneq}) is the addition of a constant function $\varphi_{X_t}$ independent of the solution multiplied by the normal vector, which cannot affect the second order terms (the normal vector only depends on first-order derivatives). Therefore, following this line of reasoning, one can apply theorems like the maximum principle (Proposition \ref{firstapplication}) and the boundary point lemma (Proposition \ref{secondapplication}) also to solutions of (\ref{chapter 4 equation}) as long as they share the same $\varphi_{X_t}$ (which is the case for reflections), as they must hold for solutions of the associated parabolic equation (\ref{parabolic version}). %A similar argument can also be used to apply the same results to solutions of equation (\ref{mccoy equation}).

Immediate consequence of the above is the fact that theorem \ref{chow gulliver} extends to solutions of equation (\ref{chapter 4 equation}):

\begin{theorem}
	Let $X:M^n\times [0,T) \rightarrow \R^{n+1}$ be a $C^2$ solution to equation (\ref{chapter 4 equation}). Then, if we can reflect $X(M^n, 0)=X_0$ strictly with respect to $\pi$, then for all $t\in [0,T)$ we can reflect $X(M^n, t)=X_t$ strictly with respect to $\pi$. \label{chow gulliver chapter 4}
\end{theorem}

The proof is word for word identical to that of theorem \ref{chow gulliver}, once one notices that it is possible to apply the maximum principle and Hopf's boundary point lemma. 

\begin{oss}\em
	There is also a simpler approach to apply the maximum principle in this specific case: in the proof of the maximum principle one considers the difference of two solutions; in principle the extra term $\phi(t) \nu$ could be different, providing an obstruction in the proof. However, reflections have the same $\phi(t)$, as the two surfaces in this case are just one the reflection of the other, and therefore this non-local term cancels out completely when considering the equation of the difference, leaving only the same terms as we had when dealing with equation (\ref{evolutioneq}). Thus, the maximum principle \ref{firstapplication} can be extended to this situation. Similarly, this also applies to Hopf's boundary point lemma \ref{secondapplication}. 
\end{oss}

As a direct consequence, all the corollaries of theorem \ref{chow gulliver} can be extended. In particular:


\begin{cor}
	Let $X:M^n\times [0,T) \rightarrow \R^{n+1}$ be a $C^2$ solution to equation (\ref{chapter 4 equation}). Then, if we can reflect $X_0$ strictly up to $(\pi_{v,C},v)$, for all $t\in [0,T)$ we can reflect $X(M, t)$ strictly up to $(\pi_{v,C},v)$. 
\end{cor}

\begin{cor}
	Let $X:M^n\times [0,T) \rightarrow \R^{n+1}$ be a $C^2$ embedded solution to equation (\ref{chapter 4 equation}). Then, if we can reflect $X_0$ strictly up to $(\pi_{v,C},v)$, for all $t\in [0,T)$ $v\notin T_xX_t$ for all $x\in X_t\cap\overline{H^+(\pi)}$. In particular,  $ X_t\cap\overline{H^+(\pi)}$ is a graph over $\pi$ for all $t\in [0,T)$.
\end{cor}


\begin{cor}
	Let $X:M^n\times [0,T) \rightarrow \R^{n+1}$ be a $C^2$ embedded solution to equation (\ref{chapter 4 equation}). There exists $C>0$ depending only on $X_0$ such that for all $t\in[0, T)$: 
	\begin{align*}
		\max_{x\in X_t} |x| - \min_{x\in X_t} |x| < C
	\end{align*}\label{new sandwich estimate}
\end{cor}
 

\begin{cor}
	Let $ X : M^n \times [0, T) \to \mathbb{R}^{n+1} $ be an embedded solution to equation (\ref{chapter 4 equation}). Then, if, for a sphere $B$, $X_0\subset B$, at all times $t \in [0, T)$ $X_t\setminus B$ is star-shaped with respect to the centre of $B$.\label{starshaped ch4}
\end{cor}

These correspond to corollaries \ref{upto}, \ref{graph}, \ref{sandwich estimate} and \ref{starshaped}, respectively.
 
\section{The solution stays inside a compact}\label{The solution stays inside a compact}

One of the results that is proven in \cite{mccoy} is that the solution to (\ref{mccoy equation}) remains inside a bounded region of euclidean space. As a final application of the technique, a simple proof of this fact for solutions to (\ref{chapter 4 equation}) is presented. The only other result we need do prove this is the well known:
\begin{theorem}[Isoperimetric inequality]
	Among all measurable sets in $\R^n$ with a given volume and $C^1$ boundary, the sphere has the smallest possible surface area.
\end{theorem} 
We show that the solution does not leave a compact ball if it has a sufficiently large radius.
\begin{comment}
First, we want to prove the following lemma: 
	
	\begin{lemma}
		Let $X:M^n\times [0,T) \rightarrow \R^{n+1}$ be a $C^2$ embedded solution to equation (\ref{chapter 4 equation}), let $\pi$ be a plane. If $X_0 \subset H^-(\pi)$, then it is not possible that $X_t \subset H^+(\pi)$ for all times $t \in [0,T)$\label{cannot cross plane ch4}
	\end{lemma}
	In other words, a solution cannot completely cross from one side of a given plane to the other side.
	\begin{proof}
		Suppose by contradiction that this happens. We can always reflect $X_t$ strictly about $\pi$, by theorem \ref{chow gulliver chapter 4}, because it does not intersect $X_0$. This implies that $X_t^\pi\cap H^-(\pi) \subset \mathrm{int}(X_t)\cap  H^-(\pi)$. But $X_t \subset H^+(\pi)$  implies that $\mathrm{int}(X_t)\cap  H^-(\pi)=\emptyset$ and  $\emptyset \neq X_t^\pi \subset H^-(\pi)$, a contradiction
	\end{proof}
	Finally, we show that the solution does not leave a compact ball if it has a sufficiently large radius. 
\end{comment}
\begin{proposition}
	Let $X:M^n\times [0,T) \rightarrow \R^{n+1}$ be a $C^2$ embedded solution to equation (\ref{chapter 4 equation}). There exists $C>0$ depending only on $X_0$ such that for all $t\in[0, T)$ such that 
	\begin{align*}
		X_t \subset B_C(0)
	\end{align*}
\end{proposition}
\begin{proof}
	The thesis can be written as 
	\begin{align*}
		\max_{x\in X_t} |x| < C
	\end{align*}
	In light of corollary \ref{new sandwich estimate}, it suffices to prove that 
	\begin{align*}
		\min_{x\in X_t} |x| < K
	\end{align*}
	as in that case 
	\begin{align*}
		\max_{x\in X_t} |x| < C + \min_{x\in X_t} |x| < C+K
	\end{align*}
	Like in the proof of corollary \ref{new sandwich estimate}, assume $X_0\subset B_K(0)$. We may take, without loss of generality, $K$ such that the surface area of $\partial B_K(0)$ is greater than the surface area of $X_0$. 
	Suppose that, at some time $t$, $\min_{x\in X_t} |x| > K$. Then, all the points in $X_t$ are outside $B_K(0)$.  By corollary \ref{starshaped ch4}, therefore, the hypersurface is star-shaped with respect to the origin and $B_K(0)\subset \mathrm{int}(X_t)$. If we chose $\phi(t)$ such that it is an area-preserving flow, this is a contradiction, because in this case we would have found an $X_t$ with a bigger volume than the sphere $B_K(0)$ but smaller surface area. For the volume-preserving flows, instead, we can compute the derivative of the total surface-area: 
	\begin{align*}
		\frac{d}{d t}\int_{X_t} d\mu &=  \int_{X_t}  \left[- H^k + \phi(t)\right] H d\mu \\
		&= - \int_{X_t}  H^{k+1} d\mu  + \phi(t)\int_{X_t} H d\mu\\
		&= - \int_{X_t}  H^{k+1} d\mu  + \frac{1}{|X_t|}\left(\int_{X_t} H^k(x, t) \, d\mu\right)\left(\int_{X_t} H d\mu\right)
	\end{align*}
	We find that 
	\begin{align*}
		\frac{1}{|X_t|}\int_{X_t}  H^{k+1} d\mu &=\left(\frac{1}{|X_t|}\int_{X_t}  H^{k+1} d\mu\right)^{\frac{k}{k+1}}\left(\frac{1}{|X_t|}\int_{X_t}  H^{k+1} d\mu\right)^{\frac{1}{k+1}}\\ &=\left(\frac{1}{|X_t|}\int_{X_t}  \left(H^{k}\right)^{\frac{k+1}{k}} d\mu\right)^{\frac{k}{k+1}}  \left(\frac{1}{|X_t|}\int_{X_t}  H^{k+1} d\mu\right)^{\frac{1}{k+1}}\\
		&\geq \left(\frac{1}{|X_t|}\int_{X_t}  H^{k} d\mu\right) \left(\frac{1}{|X_t|}\int_{X_t}  H d\mu\right)
	\end{align*}
	by Jensen's inequality. Therefore, 
	\begin{align*}
		\frac{1}{|X_t|} \left(\int_{X_t}  H^{k} d\mu\right) \left(\int_{X_t}  H d\mu\right)&\leq \int_{X_t}  H^{k+1} d\mu\\
		\frac{d}{d t}\int_{X_t} d\mu &\leq 0
	\end{align*}
	which implies that the surface area is non-increasing, and we have a contradiction similar to the one for the area-preserving case: $X_t$ is a hypersurface containing the sphere $B_K(0)$, while having a smaller surface area than said sphere. This implies that 
	\begin{align*}
		\min_{x\in X_t} |x| < K
	\end{align*}
	and the final theorem is proven. 
\end{proof}


