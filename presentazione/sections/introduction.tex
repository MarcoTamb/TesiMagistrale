% !TeX spellcheck = it_IT
% --- Equazione di cui ci occupiamo ---
\begin{frame}{Equazione di cui ci occupiamo}{}
	\begin{block}{Flusso geometrico di cui ci occupiamo}
		$X_0$ varietà differenziabile \textit{embedded} in $\mathbb{R}^{n+1}$, la facciamo evolvere secondo
		\begin{align*}
				\frac{\partial X_t}{\partial t} = - F(\kappa_1(x), \dots , \kappa_n(x)) \nu
		\end{align*}
		dove $\nu$ è il vettore normale, $\kappa_i$ le curvature principali e $F$ una funzione simmetrica tale che 
		\begin{align*}
			\frac{\partial F}{\partial \kappa_i} > 0 \mathrm{\; for \; all } \; i=1,\dots, n
		\end{align*}
	\end{block}
	%\begin{block}<2->
		%Quello di cui voglio convincervi con questa presentazione è che le soluzioni di questo flusso diventano più \textit{tonde} e \textit{simmetriche} andando avanti nel tempo
	%\end{block}
\end{frame}


\section{Introduzione}

% --- Struttura della tesi ---
\begin{frame}{Struttura della tesi}{}

\begin{itemize}
	\item \textbf{Capitolo 1}: richiami e risultati preliminari 
	\item \textbf{Capitolo 2}: introduzione del metodo dei piani di Alexandrov in $\mathbb{R}^n$, $\mathbb{H}^n$, $S^n$
	\item \textbf{Capitolo 3}: dimostrazione del teorema di Chow-Gulliver e conseguenze in $\R^n$
	\item \textbf{Capitolo 4}: estensione del teorema di Chow-Gulliver ad $\mathbb{H}^n$ ed $S^n$ 
	\item \textbf{Capitolo 5}: flussi che preservano l'area e il volume
\end{itemize}
\end{frame}


% --- Table of contents ---
\begin{frame}
	\frametitle{Struttura della presentazione}
	\tableofcontents
\end{frame}

% --- Di cosa parliamo ---
\begin{comment}{Di cosa parliamo oggi}{}
\begin{itemize}
	\item Alexandrov Moving Planes Method
	\item Il risultato di Chow-Gulliver
	\item Alcune conseguenze e corollari
	\item Cenni sull'estensione a spazi a curvatura costante
\end{itemize}
\end{comment}



% --- Di cosa *non* parliamo ---
%\begin{frame}{Di cosa \textbf{non} parliamo oggi}{(ma che è nella tesi)}
%
	%\begin{itemize}
		%\item Dettagli e dimostrazioni (oltre a qualche cenno sul teorema principale)
		%\item Discussioni approfondite sugli spazi a curvatura costante
		%\item Flussi che preservano l'area e il volume
	%\end{itemize}
%\end{frame}

