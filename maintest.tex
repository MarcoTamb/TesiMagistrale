\documentclass[a4paper,12pt]{book}
\usepackage[utf8]{inputenc}
\usepackage[italian]{babel}
\usepackage{amsmath}
\usepackage{amsfonts}
\usepackage{amssymb}
\usepackage{amsthm}
\usepackage{graphicx}
\usepackage{hyperref}
\usepackage{caption}
\usepackage{subcaption}
\usepackage{bbm}
\usepackage{comment}
\usepackage{wrapfig}
\usepackage{bm}
\usepackage{tikz}
\usetikzlibrary{arrows, automata, backgrounds, calendar, chains, matrix, mindmap, patterns, petri, shadows, shapes.geometric, shapes.misc, spy, trees}
\usepackage[a4paper,top=3cm,bottom=3cm,left=3cm,right=3cm]{geometry}
\usepackage{setspace}
\onehalfspacing
%\usetikzlibrary{decorations.text}


\newtheorem{theor}{Teorema}[chapter]
\newtheorem{lemma}[theor]{Lemma}
\newtheorem{prop}[theor]{Proposizione}
\newtheorem{coroll}[theor]{Corollario}


\theoremstyle{definition}
\newtheorem{defn}{Definizione}[chapter]
\newtheorem{ex}{Esempio}[chapter]
%\newtheorem*{definizione-numr}{Definizione}

\theoremstyle{remark}
\newtheorem{oss}{Osservazione}[chapter]


\newenvironment{skproof}{%
	\renewcommand{\proofname}{Sketch di dimostrazione}\proof}{\endproof}
\newenvironment{necproof}{%
\renewcommand{\proofname}{Dimostrazione della necessarietà}\proof}{\endproof}
\newenvironment{sufproof}{%
\renewcommand{\proofname}{Dimostrazione della sufficienza}\proof}{\endproof}


\newcommand{\Tr}{\mathrm{Tr}}
\newcommand{\divv}{\mathrm{div}}
\newcommand{\Jac}{\mathrm{Jac}}
\newcommand{\Res}{\mathrm{Res}}
\newcommand{\ord}{\mathrm{ord}}
\newcommand{\mult}{\mathrm{mult}}
\newcommand{\mi}{\mu}
\newcommand{\Div}{\mathrm{Div}}
\newcommand{\PDiv}{\mathrm{PDiv}}
\newcommand{\KDiv}{\mathrm{KDiv}}
\newcommand{\Lone}{\mathcal{L}^{(1)}}
\newcommand{\Ll}{\mathcal{L}}
%\newcommand{\comment}{}

\author{Marco Tamburro \\ Relatrice: Paola Frediani}
\title{Il Teorema di Abel-Jacobi su \\superfici di Riemann compatte}
\date{19 Settembre 2017}

\begin{document}

\frontmatter
%\maketitle
\thispagestyle{empty}
\newgeometry{text={16cm,21.5cm}, left=0cm, right=0cm}
\begin{center}
	
	{\LARGE UNIVERSITÀ DEGLI STUDI DI ROMA TOR VERGATA
	}
	\vspace{\baselineskip}
	
	{\Large MACROAREA DI SCIENZE MATEMATICHE, FISICHE E NATURALI}
	
	\vspace{1cm}
	
	\includegraphics[height=2cm]{logo.pdf}\\
	
	\vspace{1cm}
	
	{\Large 
		LAUREA MAGISTRALE IN MATEMATICA PURA E APPLICATA
		\vspace{2\baselineskip}
		
		\vspace{1cm}\vspace{1cm}
		
		
		TITOLO\vspace{0.5\baselineskip}\\
		\textbf{\Large{\ttl}} %sostituire con il titolo della tesi in MAIUSCOLO
		\vspace{2\baselineskip}
		
		
		\vspace{1cm}
		\begin{tabular}{m{8cm} m{8cm}}
			
			\textbf{Relatore:}%scegliere la dicitura appropriata
			&\textbf{Laureando:}  %scegliere la dicitura appropriata
			\\
			Prof. %scegliere la dicitura appropriata
			Carlo Sinestrari %sostituire con nome e cognome del relatore
			&matricola: 0318523 %sostituire con numero di matricola del laureando
			\\
			&Marco Tamburro %sostituire nome e cognome del laureando
			\\
			
			%eliminare le righe seguenti se il correlatore non e' previsto
			%\textbf{Correlatore (o Correlatrice):}%scegliere la dicitura appropriata
			%&
			%\\
			%Prof. (o  Prof.ssa)%scegliere la dicitura appropriata
			%(Nome e cognome) %sostituire con nome e cognome del correlatore
			
			%&
			%\\
			%&
		\end{tabular}
		\vspace{5\baselineskip}
		
		\textbf{Anno Accademico 2023/2024} %sostituire con l'anno accademico di riferimento
	}
	
\end{center}
\restoregeometry
\newpage
\tableofcontents

\phantomsection
\chapter{Introduction}
To be confirmed

\mainmatter

\include{./TeX_files/definitions}
\include{./TeX_files/maps}
\include{./TeX_files/divisori}
\include{./TeX_files/AbelJacobi}

\backmatter
% bibliography, glossary and index would go here.

%\phantomsection

\clearpage
\begin{thebibliography}{11} 
	\bibitem{Miranda} Miranda, Rick; {\em Algebraic Curves and Riemann Surfaces},  American Mathematical Society, 1995. 
	
	\bibitem{Narasimhan} Narasimhan, Raghavan ; {\em Compact Riemann Surfaces}, Lectures in Mathematics ETH Zürich, 1992.
		
	\bibitem{Hatcher} Hatcher, Allen; \href{https://www.math.cornell.edu/~hatcher/AT/ATpage.html}{{\em Algebraic Topology}},  Cambridge University Press, 2002. 
	
	\bibitem{Maunder} Maunder, C. R. F.;  {\em Algebraic Topology}, Dover Publications, 1996. 	
	
	\bibitem{Kosniowski} Kosniowski, Czes; {\em Introduzione alla topologia algebrica}, Zanichelli, 1988.
	
	\bibitem{Ghigi} Ghigi, Alessandro; \href{http://www-dimat.unipv.it/ghigi/didattica/escissione-2016.pdf}{{\em Teorema di escissione per l’omologia singolare}}, http://www-dimat.unipv.it/ghigi/didattica/escissione-2016.pdf, 2016.
	
	\bibitem{DoCarmo} Do Carmo, Manfredo P.; {\em Differential Geometry of Curves and Surfaces}, Dover Publications, 2016.
	
	\bibitem{DoCarmo2} Do Carmo, Manfredo P.; {\em Riemannian Geometry}, Birkhauser, 1992.
	
	\bibitem{martinapap} Baker M., Norine S.; {\em Riemann-Roch and Abel-Jacobi theory on a finite graph}, Advances in Mathematics, Elsevier, 2007.
	
	

\end{thebibliography}
\addcontentsline{toc}{chapter}{Bibliografia}
%\include{./TeX_files/ringraziamenti}


\end{document}